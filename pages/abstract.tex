\chapter{\abstractname}

\noindent High-Performance Computing (HPC) continues to be a prominent field, characterized by increasingly intricate topologies and the growing integration of accelerators such as GPUs. In such complex systems, analyzing resource allocation and system states is crucial, encompassing both static infrastructure and dynamic runtime behavior to identify bottlenecks and adapt to system changes. While existing tools address static and dynamic analysis separately, sys-sage uniquely combines these capabilities, offering a broad spectrum of functions to examine components and their interconnections. Its architecture allows users to tailor and extend functionalities to specific use cases, albeit requiring C++ proficiency.   

To enhance usability and facilitate rapid prototyping for a wider audience, particularly those seeking an intuitive way to interact with sys-sage without extensive compilation, this work focuses on providing a Python interface. Python, a familiar language with a relatively gentle learning curve, is well-suited for prototyping and extending existing libraries. We analyze and compare the usability and performance characteristics of different technologies for achieving this Python binding, ultimately selecting and implementing a suitable solution. Furthermore, we introduce a new feature to the core sys-sage library in C++: XML import functionality. Comprehensive benchmarks are conducted to quantify the performance difference between the resulting Python interface and the native C++ implementation. The thesis concludes with a discussion of future work, including the integration of modified destructors within the sys-sage library to improve memory management and simplify the user experience.   


