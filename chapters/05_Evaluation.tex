\chapter{Evaluation}\label{chapter:Evaluation}

The following evaluation chapter assesses the usability and performance of the developed Python interface for \texttt{sys-sage} against its C++ counterpart. It also examines the integration of the new XML import feature. Through benchmarks and qualitative analysis, we quantify the performance impact of the Python bindings and discuss the trade-offs between user-friendliness and efficiency. The evaluation mainly focuses on the Python implementation without the modifications to the destructors of the base implementation, therefore we discuss the strengths and weaknesses of the solution with the mentioned modifications.

\section{Requirement Analysis and Performance Evaluation}
First we want to revisit the requirements of the Python implementation of sys-sage, which are described in \autoref{chapter:Approach}, followed by a performance evaluation of the Python and C++ implementations.
\subsection{Library Complexity and Dependencies}

The Python solution mirrors the C++ implementation, encompassing nearly all functionalities. The only current limitation is the handling of overloaded methods, that accept a list of components as arguments. In that case the user must use the alternative method. 
\begin{lstlisting}[language=Python, xleftmargin=4em, frame = single]
    c.GetAllChildrenByType(children_list, type) # this won't work
    # we can do this instead 
    children_list += c.GetAllChildrenByType(type)
    \end{lstlisting}

Attributes can be stored as Python attributes, and both import and export functionalities are available, with optional support for custom Python functions to parse the attributes. One limitation is that currently the user has no control over default attributes when exporting to XML. 

The basic setup, excluding optional dependencies like \texttt{intel\_pqos} and \texttt{nvidia\_mig}, requires only two dependencies: Python and pybind11

\subsection{Error Handling and Documentation}

Simple error messages have been integrated, but there is room for improvement in error handling and user feedback. Basic documentation is provided, outlining the functionality of each function.

\subsection{Future Support and Build/Installation}

Pybind11 is a well-established tool for creating bindings, supported by an active community and ongoing maintenance. CMAKE support allows users to build either the Python or C++ version of sys-sage, providing flexibility in deployment.

\subsection{User Experience}

The Python interface offers a more user-friendly experience, enabling rapid prototyping and testing. For example, scheduling processes on HPC systems can be quickly prototyped and analyzed using Python scripts. If performance becomes a bottleneck, the solution can be transitioned to C++ for optimal execution. Memory management, however, requires careful attention, as users must manually manage subtree deletion to avoid undefined behavior. This work presents improvements to destructors, though memory management remains a complex aspect.

The repository of the base implementation of \texttt{sys-sage} comes with some examples, which almost all of them could be easily translated to Python.

For he XML import function we provided an additional test file, to show that this new feature is fully functional. 

The evaluation would be incomplete without a comparison of the Python and C++ implementations in terms of performance. That's why we now take a look at some benchmarks.

\subsection{Performance Benchmarking}

\subsubsection{Setup and Methodology}

To evaluate performance, C++ benchmarks utilize the high-resolution clock, while Python benchmarks employ the \texttt{timeit} module. The testing environment consisted of an Arch Linux system with an Intel i7-9750H CPU and 15GB of RAM. \cite{python-timeit-docu}

It is important to acknowledge that precise benchmarking requires rigorous methodology, including accurate resource measurement, reliable process termination, deliberate core assignment, consideration of non-uniform memory access, avoidance of swapping, and isolation of individual runs. \cite{beyer2019reliable}

However, given the substantial performance disparities between Python and C++, the benchmarks presented herein provide a general overview. For use cases demanding high precision, adhering to established benchmarking standards is recommended.

The goal of this section is to compare the Python and C++ implementations to provide insights into the performance trade-offs associated with each language. As presented in the \autoref{chapter:Implementation} we have two different solutions for the Python implementation of sys-sage. For the performanceevaluation we will only consider the solution without destructor modifications.

\subsubsection{Basic Functionality Performance}

\begin{figure}[htpb]
    \centering
\begin{tabular}{lrrr}
\textbf{Function} & \textbf{Python (ns)} & \textbf{C++ (ns)} & \textbf{Factor} \\
\hline
Parsing HWloc & 2880354 & 1240570 & 2.32 \\
Parsing capsnuma & 4881208 & 365801 & 13.34 \\
Parsing mt4g & 71169350 & 52992073 & 1.34 \\
Getting HWloc subtree nodelist & 453509 & 14055 & 32.27 \\
Getting mt4g subtree nodelist & 26648203 & 589624 & 45.20 \\
Getting all components & 39845198 & 470893 & 84.62 \\
Getting numa max bw & 30936 & 70 & 441.94 \\
Creating new component & 17054 & 364 & 46.85 \\
\end{tabular}
\label{tab:Basic functionality performance}
\caption{Basic functionality performance}
\end{figure}

The results indicate that parsing functions exhibit comparable performance. However, get operations, particularly retrieving all components, are significantly slower in Python due to the overhead of converting \texttt{std::vector} to Python lists. Creating new components is also considerably more expensive in Python, attributed to the inherent overhead of wrapping C++ objects for Python. The benchmarks to get the datapath with the max bandwidth from a component, which involves additional logic beyond a simple API call, demonstrates a substantial performance difference, highlighting the cost of loops in Python.

\subsubsection{XML Import and Export Performance}

\begin{figure}[htpb]
    \centering
\begin{tabular}{lrrr}
\textbf{Function} & \textbf{Python (ns)} & \textbf{C++ (ns)} & \textbf{Factor} \\
\hline
\textbf{XML Import} & & & \\
Both functions & 1859323 & 387664 & 4.80 \\
Only simple & 175492 & 39347 & 4.46 \\
Only complex & 251298 & 35731 & 7.03 \\
None & 94179 & 37559 & 2.51 \\
\textbf{XML Export} & & & \\
Both functions & 209590 & 53481 & 3.92 \\
Only simple & 175346 & 50376 & 3.48 \\
Only complex & 225953 & 50914 & 4.44 \\
None & 163335 & 51095 & 3.20 \\
\end{tabular}
\label{tab:XML import and export performance}
\caption{XML Import and Export Performance}
\end{figure}

The XML import and export benchmarks, conducted with two components, where one of them has two attributes (one simple, one complex), reveal that the Python implementation with only complex attribute user functions is more costly. The C++ implementation, which directly writes to \texttt{xmlNodePtr}, is faster than the Python counterpart, which involves string conversions and parsing.

\subsubsection{Attribute Access Performance}

\begin{figure}[htpb]
    \centering
\begin{tabular}{lrrr}
\textbf{Operation} & \textbf{Python (ns)} & \textbf{C++ (ns)} & \textbf{Factor} \\
\hline
Retrieve Attribute & 4054 & 182 & 22.28 \\
Update Attribute & 4246 & 664 & 6.39 \\
\end{tabular}
\label{tab:Attribute access performance}
\caption{Attribute Access Performance}
\end{figure}

Attribute retrieval is faster in C++, while attribute updates involve additional steps (key existence check, memory freeing), resulting in higher overhead.

\subsection{Python implementation comparison with and without destructor modifications}

In \autoref{chapter:Implementation}, certain limitations of the initial implementation were discussed, particularly the lack of Python's ownership over created objects. The primary concerns involve potential memory leaks and an increased burden on the user, which deviates from typical Python application paradigms.

Consequently, an alternative approach employing modified destructors was explored to enable the Python interpreter to effectively manage memory deallocation. This obviates the need for manual object deletion by the user. However, this strategy introduces a potential issue: objects might be inadvertently deleted if they fall out of scope prematurely, even if they should persist within the topology. This scenario is most likely to occur when parsers are used to construct the topology, but the created objects are not subsequently registered on the Python side. In most practical use cases, objects are registered as soon as a function is called to retrieve a component from the topology. While this specific problem warrants future attention, the overall approach of utilizing modified destructors appears more promising for seamless integration with Python's memory management.
